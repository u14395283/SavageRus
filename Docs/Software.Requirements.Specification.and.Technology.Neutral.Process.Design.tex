\documentclass[a4paper,12pt]{article}

\usepackage[utf8]{inputenc}
\usepackage{hyperref}
\usepackage{graphicx}
\usepackage{float}
\graphicspath{ {images/} }

\begin{document}

\begin{titlepage}

\newcommand{\HRule}{\rule{\linewidth}{0.5mm}} % Defines a new command for the horizontal lines, change thickness here

\center % Center everything on the page
 
%----------------------------------------------------------------------------------------
%-	HEADING SECTIONS
%----------------------------------------------------------------------------------------

\textsc{\LARGE University of Pretoria}\\[1.5cm]
\textsc{\Large COS 301 - Software Engineering}\\[0.5cm]
\textsc{\large The Savage Ru's}\\[0.5cm]

%----------------------------------------------------------------------------------------
%-	TITLE SECTION
%----------------------------------------------------------------------------------------

\HRule \\[0.4cm]
{ \huge \bfseries Software Requirements Specification and Technology Neutral Process Design}\\[0.4cm] % Title of your document
\HRule \\[1.5cm]
 
%----------------------------------------------------------------------------------------
%-	AUTHOR SECTION
%----------------------------------------------------------------------------------------

\begin{minipage}{0.4\textwidth}
\begin{flushleft} \large
\emph{Author(s):}\\
Jodan \textsc{Alberts}\\ % Your name
Mark \textsc{Klingenberg}\\
Una \textsc{Rambani}\\
Ruan \textsc{Klinkert}\\
\end{flushleft}
\end{minipage}
~
\begin{minipage}{0.4\textwidth}
\begin{flushright} \large
\emph{Student number(s):} \\
14395283\\ % Student number
14020272\\
14004489\\
14022282\\

\end{flushright}
\end{minipage}\\[4cm]


%----------------------------------------------------------------------------------------
%-	DATE SECTION
%----------------------------------------------------------------------------------------

{\large \today}\\[3cm] % Date, change the \today to a set date if you want to be precise

 
%----------------------------------------------------------------------------------------

\vfill % Fill the rest of the page with whitespace

\end{titlepage}

\newpage

\tableofcontents

\newpage

\section{Introduction}

This is the software requirements specification for the vizARD Augmented Reality application being developed for EPI-USE Labs by The Savage Ru's.

\newpage
\section{Vision}

\newpage
\subsection{Background}

It is much simpler for us to recognize patterns and make quick analysis of data if it is presented to us in visual form. A simple example for the use of such an application would be a principal at a school who is presented with the Mathematics results of a particular grade for several quarters, such an application would make it very simple for him to quickly visualize the numeric data and see the trend.
\newline
\newline
The problem at hand is that there is a lot of information to go around and so little time to process. In a society that demands us to make decisions quickly, it would be wise to have a tool that aids the decision making process by making the information easier to digest and that is what vizARD intends to do.
\newline
\newline
Potential users could range from students, researchers, people in business, managers at stores and anyone else who would like to visualize data on the go.
		

\newpage
\section{Architecture Requirements}

\subsection{Architectural Scope}

\subsection{Access Channel Requirements}

\subsection{Quality Requirements}

\subsubsection{Performance}

\subsubsection{Reliability}

\subsubsection{Scalability}

\subsubsection{Usability} 

\subsubsection{Auditability}

\subsubsection{Security}

\subsection{Integration Requirements}

\subsection{Architecture Constraints}

\subsection{Use case prioritization}
	\subsubsection{Critical}
		\begin{itemize}
			\item Taking a picture
			\item OCR (Optical Character Recognition)
			\item Automatic Graph Suggestion Algorithm
			\item Graph Generation
			\item Mapping Graph to Page
		\end{itemize}
	\subsubsection{Important}
		\begin{itemize}
			\item Live Augmented Reality Mapping
			\item Editing Graphs
			\item iOS Application
			\item Social Media Sharing
		\end{itemize}
	\subsubsection{Nice to Have}
		\begin{itemize}
			\item Opening Previous Graph
		\end{itemize}

\subsection{Use case/Services contracts}

\subsection{Required functionality}

\subsection{Process specifications}

\subsection{Domain Model}

\newpage
\section{Software Architecture}

\subsection{Architectural Patterns or Styles}
	
\subsection{Architectural Tactics or Strategies}

\subsection{Use of Reference Architectures and Frameworks}

\subsubsection{Web 2.0 Reference Architecture}

\subsection{Access and Integration Channels}

\subsection{Technologies}

\newpage
\section{Open Issues}


\end{document}
