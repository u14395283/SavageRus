\documentclass[a4paper,12pt]{article}
\usepackage[utf8]{inputenc}
\usepackage{hyperref}
\usepackage{graphicx}
\usepackage{float}
\graphicspath{ {images/} }

\usepackage{listings}
\usepackage{color}
\usepackage{adjustbox}

\definecolor{dkgreen}{rgb}{0,0.6,0}
\definecolor{gray}{rgb}{0.5,0.5,0.5}
\definecolor{mauve}{rgb}{0.58,0,0.82}

\lstset{frame=tb,
  language=[Sharp]C,
  aboveskip=3mm,
  belowskip=3mm,
  showstringspaces=false,
  columns=flexible,
  basicstyle={\small\ttfamily},
  numbers=none,
  numberstyle=\tiny\color{gray},
  keywordstyle=\color{blue},
  commentstyle=\color{dkgreen},
  stringstyle=\color{mauve},
  breaklines=true,
  breakatwhitespace=true,
  tabsize=3
}

\begin{document}

\begin{titlepage}

\newcommand{\HRule}{\rule{\linewidth}{0.5mm}} % Defines a new command for the horizontal lines, change thickness here

\center % Center everything on the page
 
%----------------------------------------------------------------------------------------
%-	HEADING SECTIONS
%----------------------------------------------------------------------------------------
\begin{center}
	\includegraphics[width=7cm]{../../Images/SavageRus.png}
\end{center}	
\vfill
\textsc{\LARGE University of Pretoria}\\[1.5cm]
\textsc{\Large COS 301 - Software Engineering}\\[0.5cm]
\textsc{\large The Savage Ru's}\\[0.5cm]

%----------------------------------------------------------------------------------------
%-	TITLE SECTION
%----------------------------------------------------------------------------------------

\HRule \\[0.4cm]
{ \huge \bfseries VizARD Testing Document}\\[0.4cm] % Title of your document
{\large \today}
\HRule \\[1.5cm]
 
%----------------------------------------------------------------------------------------
%-	AUTHOR SECTION
%----------------------------------------------------------------------------------------

\begin{minipage}{0.4\textwidth}
\begin{flushleft} \large
\emph{Author(s):}\\
Jodan \textsc{Alberts}\\ % Your name
Mark \textsc{Klingenberg}\\
Una \textsc{Rambani}\\
Ruan \textsc{Klinkert}\\
\end{flushleft}
\end{minipage}
~
\begin{minipage}{0.4\textwidth}
\begin{flushright} \large
\emph{Student number(s):} \\
14395283\\ % Student number
14020272\\
14004489\\
14022282\\

\end{flushright}
\end{minipage}\\[4cm]

 % Date, change the \today to a set date if you want to be precise

 
%----------------------------------------------------------------------------------------

\vfill % Fill the rest of the page with whitespace

\end{titlepage}

\newpage

\tableofcontents

\newpage

\section{Introduction}

\subsection{Purpose}
This is the main testing document for the VizARD Augmented Reality application being developed for EPI-USE Labs by The Savage Ru's.

VizARD is a mobile application which will allow a user to take a picture of tabulated data and then view, automatically generated, 3D graphs of the data projected onto the document of which the image was taken.

For the system we are developing, test drive development (TDD) is essential, mainly due to the uncertain aspects of new technologies being used in the system. By following a test driven cycle we can ensure that there is constant change and improvement upon the system. By incorporating unit-tests concurrently with the development cycle we can minimize time taken during the integration stage for fixing functional errors which are not related to the integration itself. TDD also allows us to almost always have at least one stable version of the system available - since all the integrated code has already been tested and meets its requirements.

The document you are reading provides an overview of the testing process we follow. It explains the strategies we've followed and the results we received in so doing.

\subsection{Scope}
\subsubsection{Requirements to Test}
We list the goals as well as some info on the type of testing for each requirement below.
\begin{itemize}
\item Usability: this is tested mainly with user-acceptance tests as things such as the stability of the graph are not measurable empirically. \\ \\
	Graphs should be:
	\begin{itemize}
		\item visible on the target table,
		\item rotated towards the viewer,
		\item scaled correctly,
		\item stable and not shaking.
	\end{itemize}
\item Performance: here we have set empirical goals for performance which we are aimed at.
	\begin{itemize}
		\item Graph generation time should not exceed 5 seconds.
		\item Application must respond to touch within 1 second of input.
		\item OCR data should be returned in no more than 5 seconds from the picture being taken.
	\end{itemize}
\item Reliability: this is also easily measured with empirical data.
	\begin{itemize}
		\item Image resolution should be high enough for consistent OCR results. Specifically no smaller than 600px on the longest side.
	\end{itemize}
\end{itemize}

Testing is done in the Unity Test Tools suite. This suite includes frameworks for Unit Testing through NUnit and a control panel from which one can run tests and receive results. This method of unit testing simplifies the process but has its limitations in our context since it does not cater to integration testing well. Partly due to the nature of the application in that it is run on Unity3D - which is not perfectly suited to running conventional applications but rather games. 

Our strategy also suffers slightly in our inability to conduct proper stress testing - where the OCR server is concerned. Thus some tests may succeed under low traffic but fail under load. Furthermore our user acceptance tests are inevitably subjective, but in this regard the client will have the final say.

To counteract for our testing limitations we are making certain assumptions with regards to the hardware the server is running on (we assume that the OCR makes optimal use of the hardware). We also assume that users are running the app on mid-range to high-end (with minimum 1gb RAM and a dual-core CPU) Android devices.
\subsection{Test Environment}
\begin{itemize}
\item Hardware:
	\begin{itemize}
		\item PC running Windows 10 and an Intel chipset with an integrated 720p webcam.
		\item Wileyfox Storm handset running Android 6.0.1.
		\item LG G4 handset running Android 6.
	\end{itemize}
\item Network:
	\begin{itemize}
		\item Internet connection of no slower than 2 mbps.
	\end{itemize}
\item Programming Languages: 
	\begin{itemize}
		\item C\# (for tests and functions)
		\item C++ (Unity3D is written in this)
	\end{itemize}
\item Testing Frameworks:
	\begin{itemize}
		\item  Unity Test Tools (including NUnit)
	\end{itemize}
\item Coding Environment:
	\begin{itemize}
		\item Mono Develop
	\end{itemize}
\item Operating Systems:
	\begin{itemize}
		\item Android OS (no older than Android 5.1)
		\item Microsoft Windows (no older than Windows 7 SP1)
		\item iOS
	\end{itemize}
\end{itemize}

\subsection{Assumptions and Dependencies}
\begin{itemize}
	\item Constant, stable, internet connection.
	\item Internet speeds of no less than 2 mbps.
	\item Server hardware powerful enough to handle traffic without unintended delays.
	\item Cellphone camera with resolution exceeding that of the devices screen.
	\item The application is \textbf{dependent} on the remote OCR server and cannot function if a connection to it cannot be established.
\end{itemize}
\section{Test Items}

\begin{itemize}
\subsection{Display Graph (Augmented Reality}
\item ModelInstantiator.cs
\item Graph.cs
\end{itemize}

\subsection{Data Gathering (Optical Character Recognition)}
\begin{itemize}
\item StringManipulation.cs
\item ImageManipulation.cs
\item UDTEventHandler.cs
\end{itemize}
\section{Functional Features to be Tested}
\subsection{Taking a picture}
Image capture is the first fundamental function which VizARD must have. Testing for this feature is simple. The app is prompted to take a picture, the picture is printed to a location on the host machine, and finally the picture is inspected to confirm its accuracy.

\subsection{OCR and Data Capture}
Data analysis is done remotely by OpenOCR and Tesseract. Sending, receiving and interpreting data is done natively in the app. Specifically, the features concerned here are:
\begin{itemize}
	\item encoding the image for the server,
	\item sending a request,
	\item receiving a response,
	\item sanitizing results,
	\item and interpreting the results.
\end{itemize}

\subsection{Generating a 3D model from captured data}
Graphs are generated from the global dataset object which stores the captured data. There are several types of graphs which may be generated, and are tested, listed below:
\begin{itemize}
	\item Bar graph
	\item Line graph
	\item Pie chart
	\item Scatter plot
\end{itemize}

The objective of the tests was to pass a variety of values to the graph generation classes, and test if the graph is successfully created.\\

\subsection{Viewing the 3D graph in Augmented Reality}
Viewing is tested manually by generating a 3D Model with mock-data and using a pre-defined AR target for it to project to. Starting with a well-lit, and clearly defined target, and then descending in quality until tracking is no longer possible. Since Vuforia is used for the AR mapping, testing and modification is limited. Simply put, our findings show that targets which are difficult to track are not easily parsed by the OCR and, as such, are not a significant concern in its current configuration.

\section{Test Cases}
\subsection{Graph Generation Test Cases}
\subsubsection{Bar, Point and Line Graphs}
\begin{itemize}
	\item Objective: The purpose of this test is to determine whether the graph generation occurs for all correct input values.
\end{itemize}

\subsubsection{Pie chart}
\begin{itemize}
	\item Objective: The purpose of this test is to determine whether pie chart models are generated successfully. \\\
\end{itemize} 

\textbf{Input Variables: }
		\begin{itemize}
			\item Values = Two dimensional array of floats representing all graph values
			\item Categories = string array with names of each category (x-axis)
			\item Series = string array with names of each series (z-axis)
			\item Categories\_title = title for categories
			\item Series\_title = title for series
			\item Series\_Count = number of series
			\item Category\_Count = number of categories
		\end{itemize}
\textbf{Pre-checks:}\\
If a pre-check fails, the graph will be adjusted but will still be drawn.

	\begin{itemize}
		\item The Length of Categories must be null or equal to Category\_Count \\
		\textbf{Action:} The labels will not be drawn.
		\item The Length of Series must be null or equal to Series\_Count \\
		\textbf{Action:} The labels will not be drawn.
		\item Categories\_title may be null\\
		\textbf{Action:} irrelevant. 
		\item Series\_title may be null\\
		\textbf{Action:} irrelevant. 
		\item Height and Width of Values must be equal to Series\_Count and Category\_Count respectively\\
		\textbf{Action:} Series\_Count and Category\_Count will be adjusted to the height and width of Values.
	\end{itemize}
	
\subsection{Optical Character Recognition Test Cases}
\subsubsection{String Sanitization}
\begin{itemize}
	\item Objective: The purpose of this test is to determine whether the OCR sanitization function successfully checks for incorrect characters.
	\item Input: several images of different tables.
	\item Outcome: > 70\% accuracy on input data.
\end{itemize}

\section{Item Pass/Fail Criteria}
\begin{itemize}
	\item Graph Generation
	\begin{itemize}
		\item Pass: correct 3D graph is generated.
		\item Fail: model has noticeable discrepancies.
	\end{itemize}
	
	\item Optical Character Recognition
	\begin{itemize}
		\item Pass: returned string contains no non-digit characters.
		\item Fail: there are non-digit characters in the returned string.
	\end{itemize}
\end{itemize}

\newpage
\section{Test Results}
\subsection{Overview Test Results}
During unit testing the majority of our tests passed. In all cases OCR reacted as expected and graphs were generated when they were expected. We used NUnit for unit testing but we were not able to automate integration testing due to the visual nature of the application and the uniqueness of the technology stack we are using. NUnit assisted greatly in executing unit tests and returning concise results. It allowed us to run a large number of tests efficiently and easily with repeatable circumstances.

\subsection{Graph Generation Test Cases}
\textbf{Test Data}
\begin{table}[ht]
\centering
\label{my-label}
\begin{tabular}{|l|l|l|l|l|l|}
\hline
Case & Values                                                                                           & Categories                                                                             & Series                                                         & Categories\_Count & Series\_Count \\ \hline
1    & {[}1,2,3,4{]}                                                                                    & {[}{]}                                                                                 & {[}{]}                                                         & 1                 & 4             \\ \hline
2    & \begin{tabular}[c]{@{}l@{}}{[}1,2,3,4{]}.\\ {[}5,6,7,8{]}\end{tabular}                           & \begin{tabular}[c]{@{}l@{}}{[}"1","2","3","4"{]},\\ {[}"5","6","7","8"{]}\end{tabular} & {[}{]}                                                         & 2                 & 4             \\ \hline
3    & \begin{tabular}[c]{@{}l@{}}{[}1,2,3,4{]}.\\ {[}5,6,7,8{]}\end{tabular}                           & {[}{]}                                                                                 & \begin{tabular}[c]{@{}l@{}}{[}"1"{]},\\ {[}"2"{]}\end{tabular} & 2                 & 4             \\ \hline
4    & \begin{tabular}[c]{@{}l@{}}{[}1,2,3,4{]}.\\ {[}5,6,7,8{]}\end{tabular}                           & \begin{tabular}[c]{@{}l@{}}{[}"1","2","3","4"{]},\\ {[}"5","6","7","8"{]}\end{tabular} & \begin{tabular}[c]{@{}l@{}}{[}"1"{]},\\ {[}"2"{]}\end{tabular} & 2                 & 4             \\ \hline
5    & \begin{tabular}[c]{@{}l@{}}{[}-1,-2{]},\\ {[}-3,-4{]},\\ {[}-5,-6{]}.\\ {[}-7,-8{]}\end{tabular} & {[}{]}                                                                                 & {[}{]}                                                         & 4                 & 2             \\ \hline
6    & \begin{tabular}[c]{@{}l@{}}{[}0.1,0.2{]},\\ {[}0.3,0.4{]}\end{tabular}                           & {[}{]}                                                                                 & {[}{]}                                                         & 2                 & 2             \\ \hline
7    & {[}{]}                                                                                           & {[}{]}                                                                                 & {[}{]}                                                         & 0                 & 0             \\ \hline
8    & \begin{tabular}[c]{@{}l@{}}{[}1,2,3,4{]},\\ {[}5,6,7,8{]}\end{tabular}                           & {[}{]}                                                                                 & {[}{]}                                                         & 0                 & 0             \\ \hline
9    & \begin{tabular}[c]{@{}l@{}}{[}1,2,3,4{]},\\ {[}5,6,7,8{]}\end{tabular}                           & {[}{]}                                                                                 & {[}{]}                                                         & 100               & 100           \\ \hline
\end{tabular}
\end{table}
\newpage
\subsubsection{Test Case 1}
\begin{itemize}
	\item Test passed
	\item This test case was meant to test if the graph would be drawn when no titles are given.
\end{itemize}
\subsubsection{Test Case 2}
\begin{itemize}
	\item Test passed
	\item This test case was meant to test if the graph would be drawn if titles for each category is given.
\end{itemize}
\subsubsection{Test Case 3}
\begin{itemize}
	\item Test passed
	\item This test case was meant to test if the graph would be drawn if titles for each series is given.
\end{itemize}
\subsubsection{Test Case 4}
\begin{itemize}
	\item Test passed
	\item This test case was meant to test if the graph would be drawn when titles are given for both categories and series.
\end{itemize}
\subsubsection{Test Case 5}
\begin{itemize}
	\item Test passed
	\item This test case was meant to test if the graph would be drawn if negative values were given as the data.
\end{itemize}
\subsubsection{Test Case 6}
\begin{itemize}
	\item Test passed
	\item This test case was meant to test if the graph would be drawn if float values were given as the data.
\end{itemize}
\subsubsection{Test Case 7}
\begin{itemize}
	\item Test passed
	\item This test case was meant to test if the graph would be drawn if no data was given.
\end{itemize}
\subsubsection{Test Case 8}
\begin{itemize}
	\item Test passed
	\item This test case was meant to test if the graph would be drawn if the series and category counts were less than the size of the dataset.
\end{itemize}
\subsubsection{Test Case 9}
\begin{itemize}
	\item Test passed
	\item This test case was meant to test if the graph would be drawn if the series and category counts were greater than the size of the dataset.
\end{itemize}

\subsection{OCR Test Cases}	
\begin{table}[h]
\centering

\label{my-label}
	\begin{adjustbox}{width=1\textwidth}
	\begin{tabular}{|l|l|l|l|}
	\hline
		\textbf{Input}         & \textbf{Returned}       & \textbf{Expected}       & \textbf{Result} \\ \hline
		"BOo\%EIlAHGbTJSMQRZCc" & "800811446677544912200" & "800811446677544912200" & Pass            \\ \hline
		"-Bo01"                 & "-8001"                 & "-8001"                 & Pass            \\ \hline
		"Bo01-"                 & "8001-"                 & "8001-"                 & Pass            \\ \hline
		"Bo-01"                 & "8001"                  & "8001"                  & Pass           \\ \hline 
	\end{tabular}
	\end{adjustbox}
	\caption{Helpers.sanitizeString Unit Test Results}		
\end{table}

\end{document}
